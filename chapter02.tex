\begin{flushleft}
La vulnerabilità \textbf{CVE-2020-24656} riguarda un attacco \textbf{XXE} (XML External Entity) che è possibile condurre ai danni di Maltego. XML External Entity è una vulnerabilità che consente a un utente malintenzionato di interferire con l'elaborazione di dati XML da parte di un'applicazione. Spesso consente a un utente malintenzionato di visualizzare i file sul filesystem del server delle applicazioni e di interagire con qualsiasi sistema di back-end o esterno a cui l'applicazione stessa può accedere. Alcune applicazioni utilizzano il formato XML per trasmettere i dati tra il browser e il web server. Le applicazioni che eseguono questa operazione utilizzano praticamente sempre una libreria standard o un'API della piattaforma per elaborare i dati XML sul server. Le vulnerabilità XXE sorgono perché il linguaggio XML contiene varie funzionalità potenzialmente pericolose e i parser standard supportano queste funzionalità anche se non sono normalmente utilizzate dall'applicazione. Questa vulnerabilità riguarda le versioni 4.2.11 e precedenti di Maltego. L'attacco permette di elaborare i file MTZ (contenenti le informazioni di configurazione) e MTGL (contenenti le informazioni sul grafo), in maniera tale da ottenere certe conseguenze, come ad esempio information disclosure, esecuzione di codice attraverso SSRF e denial of service. I file MTGL e MTZ sono dei file proprietari di Maltego, molto simili ai file ZIP (anch'essi sono in formato compresso). In questo particolare attacco, la vulnerabilità permette ad un attaccante di esfiltrare i file locali dal computer della vittima. A causa del fatto che i file MTGL e MTZ sono abbastanza spesso condivisi tra collaboratori e terze parti, le possibilità che qualcuno cada vittima di questa vulnerabilità sono relativamente alte.
\end{flushleft}