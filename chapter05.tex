\begin{flushleft}
Dalla versione 4.2.12 non è più possibile sfruttare questa vulnerabilità. Questo mette in luce due aspetti:

\begin{itemize}
    \item è importante tenere sotto costante aggiornamento i propri sistemi software, spesso è possibile risolvere una vulnerabilità semplicemente scaricando l'ultima versione del prodotto software utilizzato.
    
    \item attraverso l'uso di tecniche di ingegneria sociale, l'attaccante potrebbe ingannare la vittima, facendole credere di essere un membro del team di progetto che vuole condividere un file MTGL/MTZ. Se la vittima si fida e accetta di scaricare e aprire il file, allora l'attacco ha successo. Questo mette in luce come un'attacco si basi sia su una componente tecnica (i tecnicismi dell'attacco), sia su una componente umana (le tecniche di ingegneria sociale adoperate). \'E fondamentale quindi riuscire a capire, nel limite del possibile, la vera identità del proprio interlocutore, ed educare il proprio team circa i rischi legati all'uso dell'ingegneria sociale e ai modi con cui ci si può difendere da quest'ultima.
    
\end{itemize}

\end{flushleft}