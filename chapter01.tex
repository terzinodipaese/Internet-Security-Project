\begin{flushleft}
Con il termine \textbf{OSINT} (Open Source Intelligence), si intende l'attività di raccolta di informazioni pubblicamente disponibili e quindi libere (da qui il significato di open source) con l'intento di perseguire un certo obiettivo. L'OSINT è diventata un'attività fondamentale, soprattutto grazie all'enorme quantità di dati che viene prodotta, dati che rappresentano una fonte di ricchezza sia in ambito business che non. Bisogna prendere coscienza di questo genere di attività, dal momento che molti attacchi sono frutto dello sfruttamento di falle di cui si può venire a conoscenza semplicemente applicando delle tecniche di open source intelligence. Ecco perchè è importante capire se un'organizzazione espone delle informazioni sensibili che potrebbero di fatto renderla debole. Le fonti da cui si traggono le informazioni sono le più svariate: media come TV, giornali, radio, riviste, ma anche social media, blog, siti web, pubblicazione accademiche, basi di dati, report governativi, record whois, record DNS. Le informazioni che è possibile raccogliere sono di tanti tipi: indirizzi IP, indirizzi URL, email, informazioni su persone fisiche. L'OSINT è un'attività che viene svolta da molti soggetti differenti, e l'obiettivo ultimo dell'OSINT cambia di conseguenza. I principali soggetti che praticano OSINT sono:


\begin{itemize}
    \item i \textbf{penetration tester}, che vogliono scoprire quali informazioni sono pubblicamente disponibili e che un'attaccante potrebbe sfruttare a proprio vantaggio per studiare la vittima e condurre un'attacco
    \item le \textbf{forze dell'ordine}, che devono perseguire reati informatici come il cyberbullismo, e a cui interessa conoscere il criminale che sta perpetrando il reato per poterlo poi perseguire penalmente
    \item le \textbf{agenzie di intelligence statali}, che svolgono attività di spionaggio dentro e fuori i confini nazionali
    \item i \textbf{black hat}, che prima di attaccare la vittima la studiano, facendo attività di information gathering, detta anche reconaissance, allo scopo di scoprire i punti deboli del target. Chiaramente, più informazioni un'attaccante trova, migliore è l'idea che l'attaccante si fa della vittima 
\end{itemize}


Passando alle tecniche di open source intelligence, è bene sottolineare che è sconsigliato raccogliere qualsiasi informazione sul proprio target, dato che il volume di informazioni che ci troveremmo a gestire sarebbe troppo elevato e questo risulterebbe controproducente. Bisogna capire invece lo specifico aspetto che ci interessa approfondire della vittima, e raccogliere informazioni solo su quello. Inoltre è bene fare uso di tool e programmi che permettano di automatizzare la raccolta delle informazioni, in modo da essere più veloci ed efficaci nel raccogliere e catalogare i dati. Parlando degli strumenti per fare OSINT, quello che verrà trattato in questa sede è \textbf{Maltego}, prodotto dalla software house sudafricana Paterva, che permette di fare OSINT in maniera automatizzata, e consente di creare dei grafici basati su nodi, come dei grafi, in modo da avere una rappresentazione grafica dei dati raccolti. Maltego offre la possibilità di connettere facilmente dati e funzionalità da diverse fonti utilizzando Transforms. Tramite il Transform Hub, si possono collegare i dati di oltre 30 partner di dati, una varietà di fonti pubbliche (OSINT) e i propri dati. Esistono tante versioni di Maltego, più o meno gratuite e con una diversa ricchezza di funzionalità. La versione che verrà utilizzata è la Community Edition.
\end{flushleft}